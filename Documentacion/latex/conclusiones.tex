\chapter{Conclusiones}
\label{chap:conclusiones}

En este \acs{TFG} se ha desarrollado un Sistema de Logging Geoposicional en Tiempo Real para sistemas Empotrados, consistente en dos tareas que leen datos del GPS y los insertan en memoria respectivamente.\\

El primero de los retos afrontados ha sido la instalación y configuración de las herramientas de desarrollo, debido a que no existen herramientas que integren todas las funcionalidades requeridas. Como se ha visto en el apartado \ref{tecUsadas} se han utilizado varias herramientas para poder trabajar con el sistema.\\

Como las herramientas utilizadas son independientes entre sí y desarrolladas y mantenidas por equipos diferentes surgen diversas incompatibilidades cuando aparecen versiones nuevas, por lo que se ha tenido que comprobar cuáles eran las versiones adecuadas de cada una de las diversas herramientas.\\

El mayor reto afrontado, sin embargo, ha sido el tener que implementar un Sistema de Archivos de desarrollo propio para poder evitar las restricciones de la memoria FLASH del sistema.\\

Se puede observar que la complejidad del desarrollo del sistema de control de un sistema empotrado no viene tanto del hecho de que realice una tarea compleja sino de las restricciones inherentes al mismo, como pueden ser la capacidad de la memoria, su arquitectura física interna, o la capacidad de cómputo.\\

Durante este trabajo se ha visto que la utilización de sistemas ampliamente probados y aceptados en la industria, como puede ser el sistema de archivos FAT, no siempre es lo más conveniente y, de hecho, puede no ser posible aplicarlo, teniendo que diseñar alternativas más adecuadas aunque se alejen de cualquier estándar definido.\\

Como se ha visto las restricciones del sistema hacen único el problema y complican la reutilización del sistema desarrollado en proyectos diferentes.\\

El sistema desarrollado puede cambiar fácilmente el tamaño de la memoria asignado para el sistema de archivos y el sistema de logging cambiado algunos de sus parámetros de configuración. Esto es destacable ya que la granularidad temporal de los datos recogidos puede hacer inviable el sistema de logging si solo se utiliza uno de los sectores físicos de la memoria FLASH, o 128 KBytes. La memoria utilizada por el sistema de archivos es ampliable hasta la utilización de toda la memoria disponible después de almacenar el binario de la aplicación, que será aproximadamente la mitad del tamaño total. \\

El sistema de archivos tal y como está planteado funciona en realidad con dos particiones, para evitar la restricción de que solo se pueda borrar la memoria a nivel de sector físico completo cambiando de partición en la que escribir cuando se llena la anterior, pero si fuera necesario, y sin ningún cambio en la implementación del sistema de archivos, se podría utilizar una única partición, aunque con la restricción de que los últimos datos guardados se perderían si no los hemos exportado antes.\\

Tras la realización del presente \acs{TFG} se ha demostrado que se han adquirido las competencias necesarias definidas por el Grado en Ingeniería Informática y la intensificación de Ingeniería de Computadores.\\

\quoteauthor{Daniel Aguado Araujo}


\chapter{Trabajo futuro}
\label{chap:trabajoFuturo}

Durante la realización de este \acs{TFG} se han visto restricciones de diseño que podrían impedir el correcto funcionamiento del sistema tal y como está planteado, tomando especialmente interés las restricciones relacionadas con la memoria del sistema.\\

El problema principal de la memoria del sistema es la capacidad de la misma y, en menor medida, su estructura física. La capacidad de la memoria, de solo 1 MByte y del cual cerca de la mitad lo ocupa la aplicación desarrollada, puede impedir realizar un sistema de logging de otros componentes diferentes del GPS, o al menos no tener una granularidad temporal suficiente para que tenga sentido realizar el seguimiento de los datos.\\

Como trabajo futuro se puede ampliar este \acs{TFG} para utilizar una memoria FLASH externa a través del puerto SPI del que dispone. A partir de ahí sería necesario evaluar de nuevo el diseño del sistema de archivos a utilizar, ya que al existir diferentes restricciones puede que sea mejor opción utilizar otro.\\
